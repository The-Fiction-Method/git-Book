
\usepackage[default]{opensans}
% \usepackage{fourier}\usepackage[T1]{fontenc}%	a serif font that looks okay to me

\setlength{\parindent}{2em}
%	adds an indent to the paragraphs

\usepackage{geometry}	%	allows the margins to be set
\geometry{	papersize={\psize}	}
%	this is what checks for special paper size
%	must be before margins are set, or the layout will be incorrect

\geometry{
	margin=0.5in,
%	bindingoffset=0.3in
}	\savegeometry{title}

\geometry{
	margin			=	0.5in,
	bindingoffset	=	0.3in,
	top				=	0.75in,
	headheight		=	1.0\baselineskip,
	headsep			=	0.5\baselineskip,
	%footsep		=	0.5in,
	footskip		=	0.5in,
	bottom			=	1.0in,
	heightrounded	=	true,
}	\savegeometry{pages}

%margin		all margines
%inner		inner page margin
%outer		outer page margin
%top		top margin
%headheight	header height within top margin
%headsep	distance from header to text
%footsep	distance from text to top of footer
%footskip	distance from text to bottom of foot
%bottom		bottom margin

\usepackage{ragged2e}
%	needed for proper centering on title and similar pages

\newcommand{\paraform}{
	\raggedright						%	left-aligned, not justified
	\setlength{\parindent}{2em}			%	set paragraph indentation
	\linespread{1.25}	\selectfont		%	increases line spacing for all paragraphs
	\setlength{\parskip}{0.25\baselineskip}	%	adds a quarter the height of a line between paragraphs
	}



\usepackage[none]{hyphenat}
%	prevents hyphenation at end of lines

%makes the title and author variables accessible elsewhere
\makeatletter
	\let\runtitle\@title
	\let\runauthor\@author
\makeatother

%\usepackage{graphicx}	%	powerful package for adding images/graphics


\usepackage{indentfirst}	%	adds indent to first paragraph (something Latex does not normally do)
\widowpenalty=-5000		%	to prevent widowed lines, last paragraph lines on next page
\clubpenalty=50000		%	to prevent clubed or orphaned lines, first paragraph lines on previous page


\usepackage{titletoc,tocloft}
%	Table of Contents packages

\renewcommand{\contentsname}{\hfill \bfseries Table of Contents \hfill}
%	changes and centers the title of the Table of Contents

\renewcommand{\cftchapleader}{\cftdotfill{\cftdotsep}}
\renewcommand{\cftdotsep}{2}	%original separation is 4.5
%	adds dots following chapter

%\renewcommand{\cftdot}{\rule{1em}{0.4pt}}
%\renewcommand{\cftdotsep}{0}	%makes it a solid line
%	replaces dots with line

\setlength\cftbeforetoctitleskip{0ex}
\setlength\cftaftertoctitleskip{0ex}
%	removes the vertical spacing above and below the Table of Contents header

\renewcommand{\cftbeforechapskip}{0.25\baselineskip}
%	reduces the line spacing within the TOC

\setcounter{secnumdepth}{-1}
%	removes the section number from the table of contents by setting it to start at -1 instead of 0
%	this is needed to keep it from adding section numbers within the TOC

\usepackage{titlesec}
%	for customizing the section titles

%formats how the \chapter text is supposed to look

\titleformat{\chapter}[display]{\normalfont\Large\bfseries\center}{}{0.0em}{}[\stepcounter{chapter}]
%	the \stepcounter[chapter] command at the end increments the chapter counter, so LaTeX will do the chapter numbers automatically
\titlespacing*{\chapter}{0pt}{*-4}{*1}
%	reduces the space ahead of the chapter name

%\titleformat{command}[shape]{format}{label}{separation}{before code}[after code]
%\titlespacing{<command>}{<left>}{<before-sep>}{<after-sep>}

\usepackage{fancyhdr}
%	Fancy Header package

\pagestyle{fancyplain}
\fancyhf{}	% remove everything

\renewcommand{\headrulewidth}{0pt}	\renewcommand{\footrulewidth}{0pt}
%	removes lines from header and footer

\usepackage{emptypage}
%	removes headers and footers on empty pages, such as those \cleardoublepage adds

\makeatletter
\fancypagestyle{chapterpage}
{
	\fancyhf{}
	\fancyhf[HC]{\runtitle}	%	puts the book title above chapter numbers
	\fancyhf[FC]{\thepage}	%	thepage gives the page number
}
%	makes it so the first page of a chapter gives the title, but other pages identify the chapter
%	do need to set what runtitle is in each book file though

\fancypagestyle{pages}
{
	\fancyhf{}
	\fancyhf[HLE]{\textbf\leftmark}	%leftmark apparently adds the chapter
	\fancyhf[HRO]{\textbf\runtitle}	%so on every pair of pages there is chapter and title
	\fancyhf[FC]{\thepage}	%thepage gives the page number
}
%	style for other pages
\makeatother

\usepackage{imakeidx}
%	for making and configuring an index

\usepackage{xpatch}
\xpatchcmd{\theindex}{multicols}{multicols*}{}{}
\xpatchcmd{\endtheindex}{multicols}{multicols*}{}{}
%	this is to make the Index columns unbalanced

\usepackage{hyperref}
%	the package for hyperlinks and setting their color, highlight, etc.
\hypersetup{
	pdftitle	=	\runtitle,
	pdfauthor	=	\runauthor
	}
%	sets PDF metadata

\hypersetup{
	colorlinks,	%removes the box around links
	linkcolor=black,
	filecolor=black,
	urlcolor=black,
	linktoc=all	%makes the name and page numbers links
}
%	removes highlights from links and makes chapter and page number links in Table of Contents

\makeatletter
	\let\stdchapter\chapter
	\renewcommand*\chapter
	{
	\@ifstar{\starchapter}{\@dblarg\nostarchapter}}
	\newcommand*\starchapter[1]{%
		\stdchapter*{#1}
		\thispagestyle{chapterpage}
		\pagestyle{pages}
	}
	\def\nostarchapter[#1]#2
	{
		\stdchapter[{#1}]{#2}
		\thispagestyle{chapterpage}
		\pagestyle{pages}
	}
\makeatother
%	what actually identifies if it is the first page of a chapter or not
%	is what causes the first page of the chapters and the rest to have different headers

%	command to create the Title page
\newcommand{\titlePage}{%
\loadgeometry{title}	\thispagestyle{empty}	%	clears the header

\begin{Center}
\vspace*{\fill}
	\fontsize{50pt}{55pt}\selectfont
		\parbox{0.9\columnwidth}{\centering\runtitle}	\newline
		\vspace{40pt}
	\fontsize{20pt}{22pt}\selectfont
		by\newline
		\vspace{20pt}
	\fontsize{30pt}{32pt}\selectfont
		\runauthor
\vspace*{\fill}
\normalfont

\end{Center}
}

\newcommand{\Folder}[1]{	\def\folder{#1}	}	\Folder{Chapters}
%	Supports putting the chapter files in a lower folder. Default folder of "Chapters" is set

\newcommand{\Input}[1]{	\input{"\folder/#1.ltx"}	}
%	For times I am inputing a file other than a chapter, like a Dedication

\newcommand{\Chapter}[2][]{
	\def\name{#1}
	\def\file{#2}	
	
	\ifx\name\empty
		\chapter{Chapter \thechapter}
	\else
		\chapter{#1}
%		\addcontentsline{toc}{chapter}{#1}
	\fi
	
	\input{"\folder/#2.ltx"}
}

\newcommand{\Part}[1]{
	\phantomsection
	\addcontentsline{toc}{part}{#1}
}

\newcommand{\dinkus}{\bigskip\par\centerline{*~*~*}\medskip\par}	%	dinkus command